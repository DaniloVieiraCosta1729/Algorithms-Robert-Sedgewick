\documentclass[12pt]{article}

% Pacotes para configurações adicionais
\usepackage[margin=1in]{geometry} % Configuração das margens
\usepackage{fontenc} % Usar a fonte Times New Roman
\usepackage{setspace} % Espaçamento entre linhas
\usepackage{sectsty} % Formatação de títulos de seção
\usepackage{graphicx} % Suporte para inclusão de figuras
\usepackage{amsfonts}
\usepackage{amssymb}
\usepackage{amsmath} % Pacote para equações matemáticas

% Configurações adicionais
\renewcommand{\familydefault}{\rmdefault} % Definir a fonte padrão como Times New Roman
\linespread{1.5} % Espaçamento de 1.5 entre linhas
\allsectionsfont{\sffamily\bfseries} % Títulos de seção em negrito e sem serifa

\title{Lema de Euclides}
\begin{document}


\section{Lema de Euclides}
\subsection{enunciado}
Sejam $a$, $b$, $q$ e $r$ $\in \mathbb{Z}$, se $a=bq+r$ e $0\leq r < |b|$, então $mdc(a,b)=mdc(b,r)$. $$ $$
Demonstração:
Sejam $D_{a}, D_{b}, D_{r}$ os conjuntos de divisores de $a$, $b$ e $r$ respectivamente. Para provarmos que $mdc(a,b)=mdc(b,r)$ basta mostrar que $D_{a} \cap D_{b} = D_{b} \cap D_{r}$, pois, se esses conjuntos forem iguais, então os seus máximos também são iguais. 

Suponha que $d \in D_{a} \cap D_{b}$ , então $d \vert a-qb=r \Rightarrow d \in D_{b} \cap D_{r}$.

Se $d \in D_{b} \cap D_{r}$, então $d \vert bq+r=a \Rightarrow d \in D_{a} \cap D_{b}$.

Logo $D_{a} \cap D_{b} = D_{b} \cap D_{r}$ 

$\square$

\end{document}
